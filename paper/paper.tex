\documentclass[10pt,a4paper,onecolumn]{article}
\usepackage{marginnote}
\usepackage{graphicx}
\usepackage{xcolor}
\usepackage{authblk,etoolbox}
\usepackage{titlesec}
\usepackage{calc}
\usepackage{tikz}
\usepackage{hyperref}
\hypersetup{colorlinks,breaklinks,
            urlcolor=[rgb]{0.0, 0.5, 1.0},
            linkcolor=[rgb]{0.0, 0.5, 1.0}}
\usepackage{caption}
\usepackage{tcolorbox}
\usepackage{amssymb,amsmath}
\usepackage{ifxetex,ifluatex}
\usepackage{seqsplit}
\usepackage{fixltx2e} % provides \textsubscript
\usepackage[
  backend=biber,
%  style=alphabetic,
%  citestyle=numeric
]{biblatex}
\bibliography{paper.bib}


% --- Page layout -------------------------------------------------------------
\usepackage[top=3.5cm, bottom=3cm, right=1.5cm, left=1.0cm,
            headheight=2.2cm, reversemp, includemp, marginparwidth=4.5cm]{geometry}

% --- Default font ------------------------------------------------------------
% \renewcommand\familydefault{\sfdefault}

% --- Style -------------------------------------------------------------------
\renewcommand{\bibfont}{\small \sffamily}
\renewcommand{\captionfont}{\small\sffamily}
\renewcommand{\captionlabelfont}{\bfseries}

% --- Section/SubSection/SubSubSection ----------------------------------------
\titleformat{\section}
  {\normalfont\sffamily\Large\bfseries}
  {}{0pt}{}
\titleformat{\subsection}
  {\normalfont\sffamily\large\bfseries}
  {}{0pt}{}
\titleformat{\subsubsection}
  {\normalfont\sffamily\bfseries}
  {}{0pt}{}
\titleformat*{\paragraph}
  {\sffamily\normalsize}


% --- Header / Footer ---------------------------------------------------------
\usepackage{fancyhdr}
\pagestyle{fancy}
\fancyhf{}
%\renewcommand{\headrulewidth}{0.50pt}
\renewcommand{\headrulewidth}{0pt}
\fancyhead[L]{\hspace{-0.75cm}\includegraphics[width=5.5cm]{C:/Users/mingz/OneDrive/Documents/R/win-library/4.1/rticles/rmarkdown/templates/joss/resources/JOSS-logo.png}}
\fancyhead[C]{}
\fancyhead[R]{}
\renewcommand{\footrulewidth}{0.25pt}

\fancyfoot[L]{\footnotesize{\sffamily , (2021). Latentcor: An R Package
for Latent Correlation
Estimation. \textit{Journal of Open Source Software}, (), . \href{https://doi.org/}{https://doi.org/}}}


\fancyfoot[R]{\sffamily \thepage}
\makeatletter
\let\ps@plain\ps@fancy
\fancyheadoffset[L]{4.5cm}
\fancyfootoffset[L]{4.5cm}

% --- Macros ---------

\definecolor{linky}{rgb}{0.0, 0.5, 1.0}

\newtcolorbox{repobox}
   {colback=red, colframe=red!75!black,
     boxrule=0.5pt, arc=2pt, left=6pt, right=6pt, top=3pt, bottom=3pt}

\newcommand{\ExternalLink}{%
   \tikz[x=1.2ex, y=1.2ex, baseline=-0.05ex]{%
       \begin{scope}[x=1ex, y=1ex]
           \clip (-0.1,-0.1)
               --++ (-0, 1.2)
               --++ (0.6, 0)
               --++ (0, -0.6)
               --++ (0.6, 0)
               --++ (0, -1);
           \path[draw,
               line width = 0.5,
               rounded corners=0.5]
               (0,0) rectangle (1,1);
       \end{scope}
       \path[draw, line width = 0.5] (0.5, 0.5)
           -- (1, 1);
       \path[draw, line width = 0.5] (0.6, 1)
           -- (1, 1) -- (1, 0.6);
       }
   }

% --- Title / Authors ---------------------------------------------------------
% patch \maketitle so that it doesn't center
\patchcmd{\@maketitle}{center}{flushleft}{}{}
\patchcmd{\@maketitle}{center}{flushleft}{}{}
% patch \maketitle so that the font size for the title is normal
\patchcmd{\@maketitle}{\LARGE}{\LARGE\sffamily}{}{}
% patch the patch by authblk so that the author block is flush left
\def\maketitle{{%
  \renewenvironment{tabular}[2][]
    {\begin{flushleft}}
    {\end{flushleft}}
  \AB@maketitle}}
\makeatletter
\renewcommand\AB@affilsepx{ \protect\Affilfont}
%\renewcommand\AB@affilnote[1]{{\bfseries #1}\hspace{2pt}}
\renewcommand\AB@affilnote[1]{{\bfseries #1}\hspace{3pt}}
\makeatother
\renewcommand\Authfont{\sffamily\bfseries}
\renewcommand\Affilfont{\sffamily\small\mdseries}
\setlength{\affilsep}{1em}


\ifnum 0\ifxetex 1\fi\ifluatex 1\fi=0 % if pdftex
  \usepackage[T1]{fontenc}
  \usepackage[utf8]{inputenc}

\else % if luatex or xelatex
  \ifxetex
    \usepackage{mathspec}
  \else
    \usepackage{fontspec}
  \fi
  \defaultfontfeatures{Ligatures=TeX,Scale=MatchLowercase}

\fi
% use upquote if available, for straight quotes in verbatim environments
\IfFileExists{upquote.sty}{\usepackage{upquote}}{}
% use microtype if available
\IfFileExists{microtype.sty}{%
\usepackage{microtype}
\UseMicrotypeSet[protrusion]{basicmath} % disable protrusion for tt fonts
}{}

\usepackage{hyperref}
\hypersetup{unicode=true,
            pdftitle={Latentcor: An R Package for Latent Correlation Estimation},
            pdfborder={0 0 0},
            breaklinks=true}
\urlstyle{same}  % don't use monospace font for urls
\usepackage{color}
\usepackage{fancyvrb}
\newcommand{\VerbBar}{|}
\newcommand{\VERB}{\Verb[commandchars=\\\{\}]}
\DefineVerbatimEnvironment{Highlighting}{Verbatim}{commandchars=\\\{\}}
% Add ',fontsize=\small' for more characters per line
\usepackage{framed}
\definecolor{shadecolor}{RGB}{248,248,248}
\newenvironment{Shaded}{\begin{snugshade}}{\end{snugshade}}
\newcommand{\AlertTok}[1]{\textcolor[rgb]{0.94,0.16,0.16}{#1}}
\newcommand{\AnnotationTok}[1]{\textcolor[rgb]{0.56,0.35,0.01}{\textbf{\textit{#1}}}}
\newcommand{\AttributeTok}[1]{\textcolor[rgb]{0.77,0.63,0.00}{#1}}
\newcommand{\BaseNTok}[1]{\textcolor[rgb]{0.00,0.00,0.81}{#1}}
\newcommand{\BuiltInTok}[1]{#1}
\newcommand{\CharTok}[1]{\textcolor[rgb]{0.31,0.60,0.02}{#1}}
\newcommand{\CommentTok}[1]{\textcolor[rgb]{0.56,0.35,0.01}{\textit{#1}}}
\newcommand{\CommentVarTok}[1]{\textcolor[rgb]{0.56,0.35,0.01}{\textbf{\textit{#1}}}}
\newcommand{\ConstantTok}[1]{\textcolor[rgb]{0.00,0.00,0.00}{#1}}
\newcommand{\ControlFlowTok}[1]{\textcolor[rgb]{0.13,0.29,0.53}{\textbf{#1}}}
\newcommand{\DataTypeTok}[1]{\textcolor[rgb]{0.13,0.29,0.53}{#1}}
\newcommand{\DecValTok}[1]{\textcolor[rgb]{0.00,0.00,0.81}{#1}}
\newcommand{\DocumentationTok}[1]{\textcolor[rgb]{0.56,0.35,0.01}{\textbf{\textit{#1}}}}
\newcommand{\ErrorTok}[1]{\textcolor[rgb]{0.64,0.00,0.00}{\textbf{#1}}}
\newcommand{\ExtensionTok}[1]{#1}
\newcommand{\FloatTok}[1]{\textcolor[rgb]{0.00,0.00,0.81}{#1}}
\newcommand{\FunctionTok}[1]{\textcolor[rgb]{0.00,0.00,0.00}{#1}}
\newcommand{\ImportTok}[1]{#1}
\newcommand{\InformationTok}[1]{\textcolor[rgb]{0.56,0.35,0.01}{\textbf{\textit{#1}}}}
\newcommand{\KeywordTok}[1]{\textcolor[rgb]{0.13,0.29,0.53}{\textbf{#1}}}
\newcommand{\NormalTok}[1]{#1}
\newcommand{\OperatorTok}[1]{\textcolor[rgb]{0.81,0.36,0.00}{\textbf{#1}}}
\newcommand{\OtherTok}[1]{\textcolor[rgb]{0.56,0.35,0.01}{#1}}
\newcommand{\PreprocessorTok}[1]{\textcolor[rgb]{0.56,0.35,0.01}{\textit{#1}}}
\newcommand{\RegionMarkerTok}[1]{#1}
\newcommand{\SpecialCharTok}[1]{\textcolor[rgb]{0.00,0.00,0.00}{#1}}
\newcommand{\SpecialStringTok}[1]{\textcolor[rgb]{0.31,0.60,0.02}{#1}}
\newcommand{\StringTok}[1]{\textcolor[rgb]{0.31,0.60,0.02}{#1}}
\newcommand{\VariableTok}[1]{\textcolor[rgb]{0.00,0.00,0.00}{#1}}
\newcommand{\VerbatimStringTok}[1]{\textcolor[rgb]{0.31,0.60,0.02}{#1}}
\newcommand{\WarningTok}[1]{\textcolor[rgb]{0.56,0.35,0.01}{\textbf{\textit{#1}}}}
\usepackage{longtable,booktabs}
\usepackage{graphicx,grffile}
\makeatletter
\def\maxwidth{\ifdim\Gin@nat@width>\linewidth\linewidth\else\Gin@nat@width\fi}
\def\maxheight{\ifdim\Gin@nat@height>\textheight\textheight\else\Gin@nat@height\fi}
\makeatother
% Scale images if necessary, so that they will not overflow the page
% margins by default, and it is still possible to overwrite the defaults
% using explicit options in \includegraphics[width, height, ...]{}
\setkeys{Gin}{width=\maxwidth,height=\maxheight,keepaspectratio}
\IfFileExists{parskip.sty}{%
\usepackage{parskip}
}{% else
\setlength{\parindent}{0pt}
\setlength{\parskip}{6pt plus 2pt minus 1pt}
}
\setlength{\emergencystretch}{3em}  % prevent overfull lines
\providecommand{\tightlist}{%
  \setlength{\itemsep}{0pt}\setlength{\parskip}{0pt}}
\setcounter{secnumdepth}{0}
% Redefines (sub)paragraphs to behave more like sections
\ifx\paragraph\undefined\else
\let\oldparagraph\paragraph
\renewcommand{\paragraph}[1]{\oldparagraph{#1}\mbox{}}
\fi
\ifx\subparagraph\undefined\else
\let\oldsubparagraph\subparagraph
\renewcommand{\subparagraph}[1]{\oldsubparagraph{#1}\mbox{}}
\fi

% Pandoc citation processing
\newlength{\csllabelwidth}
\setlength{\csllabelwidth}{3em}
\newlength{\cslhangindent}
\setlength{\cslhangindent}{1.5em}
% for Pandoc 2.8 to 2.10.1
\newenvironment{cslreferences}%
  {}%
  {\par}
% For Pandoc 2.11+
\newenvironment{CSLReferences}[3] % #1 hanging-ident, #2 entry spacing
 {% don't indent paragraphs
  \setlength{\parindent}{0pt}
  % turn on hanging indent if param 1 is 1
  \ifodd #1 \everypar{\setlength{\hangindent}{\cslhangindent}}\ignorespaces\fi
  % set entry spacing
  \ifnum #2 > 0
  \setlength{\parskip}{#2\baselineskip}
  \fi
 }%
 {}
\usepackage{calc} % for calculating minipage widths
\newcommand{\CSLBlock}[1]{#1\hfill\break}
\newcommand{\CSLLeftMargin}[1]{\parbox[t]{\csllabelwidth}{#1}}
\newcommand{\CSLRightInline}[1]{\parbox[t]{\linewidth - \csllabelwidth}{#1}}
\newcommand{\CSLIndent}[1]{\hspace{\cslhangindent}#1}


\title{Latentcor: An R Package for Latent Correlation Estimation}

        \author[1, 2]{Mingze Huang}
          \author[2]{Irina Gaynanova}
          \author{Christian L. Muller}
    
      \affil[1]{Department of Statistics, Texas A\& M University}
      \affil[2]{Department of Economics, Texas A\& M University}
      \affil[3]{Department of Statistics, University of Munich}
  \date{\vspace{-5ex}}

\begin{document}
\maketitle

\marginpar{
  %\hrule
  \sffamily\small

  {\bfseries DOI:} \href{https://doi.org/}{\color{linky}{}}

  \vspace{2mm}

  {\bfseries Software}
  \begin{itemize}
    \setlength\itemsep{0em}
    \item \href{}{\color{linky}{Review}} \ExternalLink
    \item \href{}{\color{linky}{Repository}} \ExternalLink
    \item \href{}{\color{linky}{Archive}} \ExternalLink
  \end{itemize}

  \vspace{2mm}

  {\bfseries Submitted:} \\
  {\bfseries Published:} 

  \vspace{2mm}
  {\bfseries License}\\
  Authors of papers retain copyright and release the work under a Creative Commons Attribution 4.0 International License (\href{http://creativecommons.org/licenses/by/4.0/}{\color{linky}{CC-BY}}).
}

\hypertarget{summary}{%
\section{Summary}\label{summary}}

The R package \emph{latentcor} provides estimation for latent
correlation with mixed data types (continuous, binary, truncated and
ternary). Comparing to \emph{MixedCCA}, which estimates latent
correlation for canonical correlation analysis, our new package provides
a standalone version for latent correlation estimation. Also we add new
functionality for latent correlation between ternary/continous,
ternary/binary, ternary/truncated and ternary/ternary cases.

Compare to MixedCCA, memory footprint.

\hypertarget{statement-of-need}{%
\section{Statement of need}\label{statement-of-need}}

Currently there is no standalone package dealing with latent correlation
for mixed data type like we did in \emph{latentcor}. The R package
\emph{stats} (Team \& others, 2013) have some functionality to calculate
different type of correlations (Pearson, Kendall and Spearman). The R
package \emph{pcaPP} (Croux, Filzmoser, \& Fritz, 2013) provides a fast
calculation for Kendall's \(\tau\). The R package \emph{MixedCCA} (Yoon,
Carroll, \& Gaynanova, 2020) have functionality for latent correlation
estimation as an intermediate step for canonical correlation analysis on
mixed data. No package deal with latent correlation across mixed data
type.

\hypertarget{usage}{%
\section{Usage}\label{usage}}

\begin{longtable}[]{@{}ll@{}}
\toprule
Type & continuous \\
\midrule
\endhead
continuous & Liu, Lafferty, \& Wasserman (2009) \\
binary & Fan, Liu, Ning, \& Zou (2017) \\
truncated & Yoon, Carroll, \& Gaynanova (2020) \\
ternary & Quan, Booth, \& Wells (2018) \\
\bottomrule
\end{longtable}

\emph{Definition 1} Fan et al.~(2017) considered the problem of
estimating \(\Sigma\) for the latent Gaussian copula model based on
Kendall's \(\tau\). Given the observed data
\((X_{1j}, X_{1k}), ..., (X_{nj}, X_{nk})\) for variables \(X_{j}\) and
\(X_{k}\), Kendall's \(\tau\) is defined as \[
\hat{\tau}_{jk}=\frac{2}{n(n-1)}\sum_{1\leq i <i'\leq n} sign(X_{ij}-X_{i'j})sign(X_{ik}-X_{i'k})
\] \emph{Theorem 1} Let \(W_{1}\in\cal{R}^{p_1}\),
\(W_{2}\in\cal{R}^{p_2}\), \(W_{3}\in\cal{R}^{p_3}\),
\(W_{4}\in\cal{R}^{p_4}\) be such that
\(W=(W_{1}, W_{2}, W_{3}, W_{4})\sim NPN(0,\Sigma,f)\) with
\(p=p_{1}+p_{2}+p_{3}+p_{4}\). Let
\(X=(X_{1}, X_{2}, X_{3}, X_{4})\in\cal{R}^{p}\) satisfy \(X_{j}=W_{j}\)
for \(j=1,...,p_{1}\), \(X_{j}=I(W_{j}>c_{j})\) for
\(j=p_{1}+1,...,p_{1}+p_{2}\), \(X_{j}=I(W_{j}>c_{j})W_{j}\) for
\(j=p_{1}+p_{2}+1,...,p\) and
\(X_{j}=I(W_{j}>c_{j}^{1})+I(W_{j}>c_{j}^{2})\) with
\(\Delta_{j}=f(c_{j})\), \(\Delta_{j}^{1}=f(c_{j}^{1})\) and
\(\Delta_{j}^{2}=f(c_{j}^{2})\). The rank-based estimator of \(\Sigma\)
based on the observed \(n\) realizations of \(X\) is the matrix
\(\hat{R}\) with \(\hat{r}_{jj}=1\),
\(\hat{r}_{jk}=\hat{r}_{kj}=F^{-1}(\hat{\tau}_{jk})\) with block
structure \[
\hat{R}=\left(\begin{array}\\
F^{-1}_{CC}(\hat{\tau})\hspace{.2in} F^{-1}_{CB}(\hat{\tau})\hspace{.2in} F^{-1}_{CT}(\hat{\tau})\hspace{.2in} F^{-1}_{CN}(\hat{\tau})\\
F^{-1}_{BC}(\hat{\tau})\hspace{.2in} F^{-1}_{BB}(\hat{\tau})\hspace{.2in} F^{-1}_{BT}(\hat{\tau})\hspace{.2in} F^{-1}_{BN}(\hat{\tau})\\
F^{-1}_{TC}(\hat{\tau})\hspace{.2in} F^{-1}_{TB}(\hat{\tau})\hspace{.2in} F^{-1}_{TT}(\hat{\tau})\hspace{.2in} F^{-1}_{TN}(\hat{\tau})\\
F^{-1}_{NC}(\hat{\tau})\hspace{.2in} F^{-1}_{NB}(\hat{\tau})\hspace{.2in} F^{-1}_{NT}(\hat{\tau})\hspace{.2in} F^{-1}_{NN}(\hat{\tau})
\end{array}\right)
\] The original method is taking estimated Kendall's \(\hat{\tau}\) and
other parameters to calculate latent correlation \(\hat{r}\). Whereas
the approximated method is using multilinear interpolation to
approximate latent correlation \(\hat{r}\) via pre-calculated grid
values (Yoon, Müller, \& Gaynanova, 2021).

refer to table for reference of formula.

Table to show memory improvement compare to mixedCCA.

\begin{Shaded}
\begin{Highlighting}[]
\FunctionTok{library}\NormalTok{(latentcor)}
\DocumentationTok{\#\#\# Data setting}
\NormalTok{n }\OtherTok{\textless{}{-}} \DecValTok{1000}\NormalTok{; p1 }\OtherTok{\textless{}{-}} \DecValTok{1}\NormalTok{; p2 }\OtherTok{\textless{}{-}} \DecValTok{1} \CommentTok{\# sample size and dimensions for two datasets.}
\NormalTok{maxcancor }\OtherTok{\textless{}{-}} \FloatTok{0.9} \CommentTok{\# true canonical correlation}

\DocumentationTok{\#\#\# Correlation structure within each data set}
\FunctionTok{set.seed}\NormalTok{(}\DecValTok{0}\NormalTok{)}
\NormalTok{perm1 }\OtherTok{\textless{}{-}} \FunctionTok{sample}\NormalTok{(}\DecValTok{1}\SpecialCharTok{:}\NormalTok{p1, }\AttributeTok{size =}\NormalTok{ p1);}
\NormalTok{Sigma1 }\OtherTok{\textless{}{-}} \FunctionTok{autocor}\NormalTok{(p1, }\FloatTok{0.7}\NormalTok{)[perm1, perm1]}
\NormalTok{blockind }\OtherTok{\textless{}{-}} \FunctionTok{sample}\NormalTok{(}\DecValTok{1}\SpecialCharTok{:}\DecValTok{3}\NormalTok{, }\AttributeTok{size =}\NormalTok{ p2, }\AttributeTok{replace =} \ConstantTok{TRUE}\NormalTok{);}
\NormalTok{Sigma2 }\OtherTok{\textless{}{-}} \FunctionTok{blockcor}\NormalTok{(blockind, }\FloatTok{0.7}\NormalTok{)}
\NormalTok{mu }\OtherTok{\textless{}{-}} \FunctionTok{rbinom}\NormalTok{(p1}\SpecialCharTok{+}\NormalTok{p2, }\DecValTok{1}\NormalTok{, }\FloatTok{0.5}\NormalTok{)}

\DocumentationTok{\#\#\# true variable indices for each dataset}
\NormalTok{trueidx1 }\OtherTok{\textless{}{-}} \DecValTok{1}
\NormalTok{trueidx2 }\OtherTok{\textless{}{-}} \DecValTok{1}

\CommentTok{\# Data generation}
\NormalTok{simdata }\OtherTok{\textless{}{-}} \FunctionTok{GenerateData}\NormalTok{(}\AttributeTok{n=}\NormalTok{n, }\AttributeTok{trueidx1 =}\NormalTok{ trueidx1, }\AttributeTok{trueidx2 =}\NormalTok{ trueidx2,}
                        \AttributeTok{maxcancor =}\NormalTok{ maxcancor,}
                        \AttributeTok{Sigma1 =}\NormalTok{ Sigma1, }\AttributeTok{Sigma2 =}\NormalTok{ Sigma2,}
                        \AttributeTok{copula1 =} \StringTok{"exp"}\NormalTok{, }\AttributeTok{copula2 =} \StringTok{"cube"}\NormalTok{,}
                        \AttributeTok{muZ =}\NormalTok{ mu,}
                        \AttributeTok{type1 =} \StringTok{"binary"}\NormalTok{, }\AttributeTok{type2 =} \StringTok{"continuous"}\NormalTok{,}
                        \AttributeTok{c1 =} \FunctionTok{rep}\NormalTok{(}\DecValTok{1}\NormalTok{, p1), }\AttributeTok{c2 =}  \ConstantTok{NULL}
\NormalTok{)}
\end{Highlighting}
\end{Shaded}

\begin{verbatim}
## Warning in GenerateData(n = n, trueidx1 = trueidx1, trueidx2 = trueidx2, : Same
## threshold is applied to the all variables in the first set.
\end{verbatim}

\begin{Shaded}
\begin{Highlighting}[]
\NormalTok{X1 }\OtherTok{\textless{}{-}}\NormalTok{ simdata}\SpecialCharTok{$}\NormalTok{X1; X2 }\OtherTok{\textless{}{-}}\NormalTok{ simdata}\SpecialCharTok{$}\NormalTok{X2; Sigma12\_tt }\OtherTok{\textless{}{-}}\NormalTok{ simdata}\SpecialCharTok{$}\NormalTok{Sigma12}
\CommentTok{\# Estimate latent correlation matrix with original method}
\NormalTok{R1\_tt\_org }\OtherTok{\textless{}{-}} \FunctionTok{estR}\NormalTok{(X1, }\StringTok{"binary"}\NormalTok{, }\AttributeTok{method =} \StringTok{"original"}\NormalTok{)}
\NormalTok{R2\_tt\_org }\OtherTok{\textless{}{-}} \FunctionTok{estR}\NormalTok{(X2, }\StringTok{"continuous"}\NormalTok{, }\AttributeTok{method =} \StringTok{"original"}\NormalTok{)}
\NormalTok{R12\_tt\_org }\OtherTok{\textless{}{-}} \FunctionTok{estR}\NormalTok{(X1, }\AttributeTok{type1 =} \StringTok{"binary"}\NormalTok{, X2, }\AttributeTok{type2 =} \StringTok{"continuous"}\NormalTok{,}
                              \AttributeTok{method =} \StringTok{"original"}\NormalTok{)}\SpecialCharTok{$}\NormalTok{R12}
\CommentTok{\# Estimate latent correlation matrix with original method}
\NormalTok{R1\_tt\_org }\OtherTok{\textless{}{-}} \FunctionTok{estR}\NormalTok{(X1, }\StringTok{"binary"}\NormalTok{, }\AttributeTok{method =} \StringTok{"approx"}\NormalTok{)}
\NormalTok{R2\_tt\_org }\OtherTok{\textless{}{-}} \FunctionTok{estR}\NormalTok{(X2, }\StringTok{"continuous"}\NormalTok{, }\AttributeTok{method =} \StringTok{"approx"}\NormalTok{)}
\NormalTok{R12\_tt\_org }\OtherTok{\textless{}{-}} \FunctionTok{estR}\NormalTok{(X1, }\AttributeTok{type1 =} \StringTok{"binary"}\NormalTok{, X2, }\AttributeTok{type2 =} \StringTok{"continuous"}\NormalTok{,}
                              \AttributeTok{method =} \StringTok{"approx"}\NormalTok{)}\SpecialCharTok{$}\NormalTok{R12}
\end{Highlighting}
\end{Shaded}

\hypertarget{rendered-r-figures}{%
\section{Rendered R Figures}\label{rendered-r-figures}}

\hypertarget{references}{%
\section*{References}\label{references}}
\addcontentsline{toc}{section}{References}

\hypertarget{refs}{}
\begin{CSLReferences}{1}{0}
\leavevmode\hypertarget{ref-croux2013robust}{}%
Croux, C., Filzmoser, P., \& Fritz, H. (2013). Robust sparse principal
component analysis. \emph{Technometrics}, \emph{55}(2), 202--214.

\leavevmode\hypertarget{ref-fan2017high}{}%
Fan, J., Liu, H., Ning, Y., \& Zou, H. (2017). High dimensional
semiparametric latent graphical model for mixed data. \emph{Journal of
the Royal Statistical Society. Series B: Statistical Methodology},
\emph{79}(2), 405--421.

\leavevmode\hypertarget{ref-liu2009nonparanormal}{}%
Liu, H., Lafferty, J., \& Wasserman, L. (2009). The nonparanormal:
Semiparametric estimation of high dimensional undirected graphs.
\emph{Journal of Machine Learning Research}, \emph{10}(10).

\leavevmode\hypertarget{ref-quan2018rank}{}%
Quan, X., Booth, J. G., \& Wells, M. T. (2018). Rank-based approach for
estimating correlations in mixed ordinal data. \emph{arXiv preprint
arXiv:1809.06255}.

\leavevmode\hypertarget{ref-team2013r}{}%
Team, R. C., \& others. (2013). R: A language and environment for
statistical computing.

\leavevmode\hypertarget{ref-yoon2020sparse}{}%
Yoon, G., Carroll, R. J., \& Gaynanova, I. (2020). Sparse semiparametric
canonical correlation analysis for data of mixed types.
\emph{Biometrika}, \emph{107}(3), 609--625.

\leavevmode\hypertarget{ref-yoon2021fast}{}%
Yoon, G., Müller, C. L., \& Gaynanova, I. (2021). Fast computation of
latent correlations. \emph{Journal of Computational and Graphical
Statistics}, 1--8.

\end{CSLReferences}

\end{document}
